% Options for packages loaded elsewhere
\PassOptionsToPackage{unicode}{hyperref}
\PassOptionsToPackage{hyphens}{url}
%
\documentclass[
]{article}
\usepackage{amsmath,amssymb}
\usepackage{iftex}
\ifPDFTeX
  \usepackage[T1]{fontenc}
  \usepackage[utf8]{inputenc}
  \usepackage{textcomp} % provide euro and other symbols
\else % if luatex or xetex
  \usepackage{unicode-math} % this also loads fontspec
  \defaultfontfeatures{Scale=MatchLowercase}
  \defaultfontfeatures[\rmfamily]{Ligatures=TeX,Scale=1}
\fi
\usepackage{lmodern}
\ifPDFTeX\else
  % xetex/luatex font selection
\fi
% Use upquote if available, for straight quotes in verbatim environments
\IfFileExists{upquote.sty}{\usepackage{upquote}}{}
\IfFileExists{microtype.sty}{% use microtype if available
  \usepackage[]{microtype}
  \UseMicrotypeSet[protrusion]{basicmath} % disable protrusion for tt fonts
}{}
\makeatletter
\@ifundefined{KOMAClassName}{% if non-KOMA class
  \IfFileExists{parskip.sty}{%
    \usepackage{parskip}
  }{% else
    \setlength{\parindent}{0pt}
    \setlength{\parskip}{6pt plus 2pt minus 1pt}}
}{% if KOMA class
  \KOMAoptions{parskip=half}}
\makeatother
\usepackage{xcolor}
\usepackage[margin=1in]{geometry}
\usepackage{color}
\usepackage{fancyvrb}
\newcommand{\VerbBar}{|}
\newcommand{\VERB}{\Verb[commandchars=\\\{\}]}
\DefineVerbatimEnvironment{Highlighting}{Verbatim}{commandchars=\\\{\}}
% Add ',fontsize=\small' for more characters per line
\usepackage{framed}
\definecolor{shadecolor}{RGB}{248,248,248}
\newenvironment{Shaded}{\begin{snugshade}}{\end{snugshade}}
\newcommand{\AlertTok}[1]{\textcolor[rgb]{0.94,0.16,0.16}{#1}}
\newcommand{\AnnotationTok}[1]{\textcolor[rgb]{0.56,0.35,0.01}{\textbf{\textit{#1}}}}
\newcommand{\AttributeTok}[1]{\textcolor[rgb]{0.13,0.29,0.53}{#1}}
\newcommand{\BaseNTok}[1]{\textcolor[rgb]{0.00,0.00,0.81}{#1}}
\newcommand{\BuiltInTok}[1]{#1}
\newcommand{\CharTok}[1]{\textcolor[rgb]{0.31,0.60,0.02}{#1}}
\newcommand{\CommentTok}[1]{\textcolor[rgb]{0.56,0.35,0.01}{\textit{#1}}}
\newcommand{\CommentVarTok}[1]{\textcolor[rgb]{0.56,0.35,0.01}{\textbf{\textit{#1}}}}
\newcommand{\ConstantTok}[1]{\textcolor[rgb]{0.56,0.35,0.01}{#1}}
\newcommand{\ControlFlowTok}[1]{\textcolor[rgb]{0.13,0.29,0.53}{\textbf{#1}}}
\newcommand{\DataTypeTok}[1]{\textcolor[rgb]{0.13,0.29,0.53}{#1}}
\newcommand{\DecValTok}[1]{\textcolor[rgb]{0.00,0.00,0.81}{#1}}
\newcommand{\DocumentationTok}[1]{\textcolor[rgb]{0.56,0.35,0.01}{\textbf{\textit{#1}}}}
\newcommand{\ErrorTok}[1]{\textcolor[rgb]{0.64,0.00,0.00}{\textbf{#1}}}
\newcommand{\ExtensionTok}[1]{#1}
\newcommand{\FloatTok}[1]{\textcolor[rgb]{0.00,0.00,0.81}{#1}}
\newcommand{\FunctionTok}[1]{\textcolor[rgb]{0.13,0.29,0.53}{\textbf{#1}}}
\newcommand{\ImportTok}[1]{#1}
\newcommand{\InformationTok}[1]{\textcolor[rgb]{0.56,0.35,0.01}{\textbf{\textit{#1}}}}
\newcommand{\KeywordTok}[1]{\textcolor[rgb]{0.13,0.29,0.53}{\textbf{#1}}}
\newcommand{\NormalTok}[1]{#1}
\newcommand{\OperatorTok}[1]{\textcolor[rgb]{0.81,0.36,0.00}{\textbf{#1}}}
\newcommand{\OtherTok}[1]{\textcolor[rgb]{0.56,0.35,0.01}{#1}}
\newcommand{\PreprocessorTok}[1]{\textcolor[rgb]{0.56,0.35,0.01}{\textit{#1}}}
\newcommand{\RegionMarkerTok}[1]{#1}
\newcommand{\SpecialCharTok}[1]{\textcolor[rgb]{0.81,0.36,0.00}{\textbf{#1}}}
\newcommand{\SpecialStringTok}[1]{\textcolor[rgb]{0.31,0.60,0.02}{#1}}
\newcommand{\StringTok}[1]{\textcolor[rgb]{0.31,0.60,0.02}{#1}}
\newcommand{\VariableTok}[1]{\textcolor[rgb]{0.00,0.00,0.00}{#1}}
\newcommand{\VerbatimStringTok}[1]{\textcolor[rgb]{0.31,0.60,0.02}{#1}}
\newcommand{\WarningTok}[1]{\textcolor[rgb]{0.56,0.35,0.01}{\textbf{\textit{#1}}}}
\usepackage{longtable,booktabs,array}
\usepackage{calc} % for calculating minipage widths
% Correct order of tables after \paragraph or \subparagraph
\usepackage{etoolbox}
\makeatletter
\patchcmd\longtable{\par}{\if@noskipsec\mbox{}\fi\par}{}{}
\makeatother
% Allow footnotes in longtable head/foot
\IfFileExists{footnotehyper.sty}{\usepackage{footnotehyper}}{\usepackage{footnote}}
\makesavenoteenv{longtable}
\usepackage{graphicx}
\makeatletter
\def\maxwidth{\ifdim\Gin@nat@width>\linewidth\linewidth\else\Gin@nat@width\fi}
\def\maxheight{\ifdim\Gin@nat@height>\textheight\textheight\else\Gin@nat@height\fi}
\makeatother
% Scale images if necessary, so that they will not overflow the page
% margins by default, and it is still possible to overwrite the defaults
% using explicit options in \includegraphics[width, height, ...]{}
\setkeys{Gin}{width=\maxwidth,height=\maxheight,keepaspectratio}
% Set default figure placement to htbp
\makeatletter
\def\fps@figure{htbp}
\makeatother
\setlength{\emergencystretch}{3em} % prevent overfull lines
\providecommand{\tightlist}{%
  \setlength{\itemsep}{0pt}\setlength{\parskip}{0pt}}
\setcounter{secnumdepth}{-\maxdimen} % remove section numbering
\ifLuaTeX
  \usepackage{selnolig}  % disable illegal ligatures
\fi
\IfFileExists{bookmark.sty}{\usepackage{bookmark}}{\usepackage{hyperref}}
\IfFileExists{xurl.sty}{\usepackage{xurl}}{} % add URL line breaks if available
\urlstyle{same}
\hypersetup{
  pdftitle={HW 4},
  pdfauthor={Andrew Plantinga},
  hidelinks,
  pdfcreator={LaTeX via pandoc}}

\title{HW 4}
\author{Andrew Plantinga}
\date{2023-10-31}

\begin{document}
\maketitle

\begin{Shaded}
\begin{Highlighting}[]
\FunctionTok{library}\NormalTok{(tidyverse)}
\FunctionTok{library}\NormalTok{(nloptr)}
\FunctionTok{library}\NormalTok{(knitr)}
\end{Highlighting}
\end{Shaded}

\hypertarget{exhaustible-resource-extraction}{%
\subsection{Exhaustible Resource
Extraction}\label{exhaustible-resource-extraction}}

Suppose there is an exhaustible resource that is costlessly extracted.
The inverse demand curve for the resource is \(p_t=a-bq_t\) where
\(q_t\) is the quantity extracted in time t and a=40, b=1. The discount
rate is \(\delta=0.05\) and the initial stock of reserves is \(R_0\) =
778.1259 units. Set up a program to find the competitive market
equilibrium and show the equilibrium values of \(p_t\), \(q_t\),
\(R_t\), and \(\pi_t\) in a table. \(R_t\) is the reserves remaining in
year t and \(\pi_t\) is the discounted profits in each year (without
costs, they are just equal to \(\rho^tp_tq_t\) ). {[}Hint: it is easiest
to start with the price in the last period and work backwards through
time.{]}

\hypertarget{a.}{%
\subsubsection{A.}\label{a.}}

List the values of \(t\), \(p_t\), \(q_t\), \(R_t\), and \(\pi_t\) in a
table below. What should be the final (year T) values of \(p_T\) and
\(R_T\)? Why? Does the resource get more scarce over time? What is your
evidence?

{\textbf{The final year T is 36, \(p_t\)=40, and \(R_t\)=0. The resource
becomes more scarce over time because it is an exhaustible resource that
will eventually be depleted. As we can see in the table, both quantity
and reserves slowly deplete over time. If there are no quantities and
reserves, profits will also be 0. }}

\begin{Shaded}
\begin{Highlighting}[]
\NormalTok{exhaust\_func }\OtherTok{\textless{}{-}} \ControlFlowTok{function}\NormalTok{(a, b, c, delta, R0, T) \{}
  
\NormalTok{  t}\OtherTok{=}\FunctionTok{seq}\NormalTok{(}\DecValTok{0}\NormalTok{,T)}
  
\NormalTok{  price}\OtherTok{=}\FunctionTok{vector}\NormalTok{(}\AttributeTok{mode=}\StringTok{"numeric"}\NormalTok{, }\AttributeTok{length=}\FunctionTok{length}\NormalTok{(t))}
\NormalTok{  quantity}\OtherTok{=}\FunctionTok{vector}\NormalTok{(}\AttributeTok{mode=}\StringTok{"numeric"}\NormalTok{, }\AttributeTok{length=}\FunctionTok{length}\NormalTok{(t))}
\NormalTok{  reserves}\OtherTok{=}\FunctionTok{vector}\NormalTok{(}\AttributeTok{mode=}\StringTok{"numeric"}\NormalTok{, }\AttributeTok{length=}\FunctionTok{length}\NormalTok{(t))}
\NormalTok{  profits}\OtherTok{=}\FunctionTok{vector}\NormalTok{(}\AttributeTok{mode=}\StringTok{"numeric"}\NormalTok{, }\AttributeTok{length=}\FunctionTok{length}\NormalTok{(t))}
  
\NormalTok{  price[T}\SpecialCharTok{+}\DecValTok{1}\NormalTok{]}\OtherTok{=}\NormalTok{a}
\NormalTok{  reserves[}\DecValTok{1}\NormalTok{]}\OtherTok{=}\NormalTok{R0}
  
  \ControlFlowTok{for}\NormalTok{(i }\ControlFlowTok{in}\NormalTok{ (T}\SpecialCharTok{+}\DecValTok{1}\NormalTok{)}\SpecialCharTok{:}\DecValTok{2}\NormalTok{) \{}
\NormalTok{    quantity[i]}\OtherTok{=}\NormalTok{(a}\SpecialCharTok{{-}}\NormalTok{price[i])}\SpecialCharTok{/}\NormalTok{b}
\NormalTok{    price[i}\DecValTok{{-}1}\NormalTok{]}\OtherTok{=}\NormalTok{(price[i]}\SpecialCharTok{+}\NormalTok{c}\SpecialCharTok{*}\NormalTok{delta)}\SpecialCharTok{/}\NormalTok{(}\DecValTok{1}\SpecialCharTok{+}\NormalTok{delta)}
\NormalTok{  \}}
  
\NormalTok{  quantity[}\DecValTok{1}\NormalTok{]}\OtherTok{=}\NormalTok{(a}\SpecialCharTok{{-}}\NormalTok{price[}\DecValTok{1}\NormalTok{])}\SpecialCharTok{/}\NormalTok{b}
  \CommentTok{\#price[1]=a/(1+delta) \#leave?}
  \CommentTok{\#reserves[1]=R0}
\NormalTok{  profits[}\DecValTok{1}\NormalTok{]}\OtherTok{=}\NormalTok{price[}\DecValTok{1}\NormalTok{]}\SpecialCharTok{*}\NormalTok{quantity[}\DecValTok{1}\NormalTok{]}
  
  \ControlFlowTok{for}\NormalTok{(i }\ControlFlowTok{in} \DecValTok{2}\SpecialCharTok{:}\NormalTok{(T}\SpecialCharTok{+}\DecValTok{1}\NormalTok{)) \{}
\NormalTok{    reserves[i]}\OtherTok{=}\NormalTok{reserves[i}\DecValTok{{-}1}\NormalTok{]}\SpecialCharTok{{-}}\NormalTok{quantity[i}\DecValTok{{-}1}\NormalTok{]}
\NormalTok{    profits[i]}\OtherTok{=}\DecValTok{1}\SpecialCharTok{/}\NormalTok{(}\DecValTok{1}\SpecialCharTok{+}\NormalTok{delta)}\SpecialCharTok{*}\NormalTok{price[i]}\SpecialCharTok{*}\NormalTok{quantity[i]}
\NormalTok{  \}}
  
  \FunctionTok{return}\NormalTok{(}\FunctionTok{data.frame}\NormalTok{(}\StringTok{"time"}\OtherTok{=}\NormalTok{t, }\StringTok{"price"}\OtherTok{=}\NormalTok{price, }\StringTok{"quantity"}\OtherTok{=}\NormalTok{quantity, }\StringTok{"reserves"}\OtherTok{=}\NormalTok{reserves, }\StringTok{"profits"}\OtherTok{=}\NormalTok{profits))}
\NormalTok{\}}

\NormalTok{exhaust\_a\_df }\OtherTok{\textless{}{-}} \FunctionTok{exhaust\_func}\NormalTok{(}\AttributeTok{a=}\DecValTok{40}\NormalTok{, }\AttributeTok{b=}\DecValTok{1}\NormalTok{, }\AttributeTok{R0=}\FloatTok{778.1259}\NormalTok{, }\AttributeTok{delta=}\FloatTok{0.05}\NormalTok{, }\AttributeTok{c=}\DecValTok{0}\NormalTok{, }\AttributeTok{T=}\DecValTok{36}\NormalTok{)}

\FunctionTok{kable}\NormalTok{(exhaust\_a\_df)}
\end{Highlighting}
\end{Shaded}

\begin{longtable}[]{@{}rrrrr@{}}
\toprule\noalign{}
time & price & quantity & reserves & profits \\
\midrule\noalign{}
\endhead
\bottomrule\noalign{}
\endlastfoot
0 & 6.906297 & 33.093703 & 778.1259000 & 228.55493 \\
1 & 7.251611 & 32.748389 & 745.0321966 & 226.17008 \\
2 & 7.614192 & 32.385808 & 712.2838080 & 234.84930 \\
3 & 7.994902 & 32.005098 & 679.8980000 & 243.69296 \\
4 & 8.394647 & 31.605353 & 647.8929016 & 252.68169 \\
5 & 8.814379 & 31.185621 & 616.2875482 & 261.79227 \\
6 & 9.255098 & 30.744902 & 585.1019272 & 270.99722 \\
7 & 9.717853 & 30.282147 & 554.3570252 & 280.26424 \\
8 & 10.203746 & 29.796254 & 524.0748780 & 289.55562 \\
9 & 10.713933 & 29.286067 & 494.2786235 & 298.82758 \\
10 & 11.249629 & 28.750371 & 464.9925563 & 308.02954 \\
11 & 11.812111 & 28.187889 & 436.2421857 & 317.10331 \\
12 & 12.402716 & 27.597284 & 408.0542965 & 325.98217 \\
13 & 13.022852 & 26.977148 & 380.4570129 & 334.58991 \\
14 & 13.673995 & 26.326005 & 353.4798652 & 342.83967 \\
15 & 14.357695 & 25.642305 & 327.1538600 & 350.63275 \\
16 & 15.075579 & 24.924421 & 301.5115546 & 357.85722 \\
17 & 15.829358 & 24.170642 & 276.5871339 & 364.38643 \\
18 & 16.620826 & 23.379174 & 252.4164922 & 370.07732 \\
19 & 17.451867 & 22.548133 & 229.0373184 & 374.76859 \\
20 & 18.324461 & 21.675539 & 206.4891859 & 378.27864 \\
21 & 19.240684 & 20.759316 & 184.8136468 & 380.40328 \\
22 & 20.202718 & 19.797282 & 164.0543307 & 380.91324 \\
23 & 21.212854 & 18.787146 & 144.2570488 & 379.55141 \\
24 & 22.273497 & 17.726503 & 125.4699028 & 376.02973 \\
25 & 23.387172 & 16.612828 & 107.7433996 & 370.02578 \\
26 & 24.556530 & 15.443470 & 91.1305711 & 361.17908 \\
27 & 25.784357 & 14.215643 & 75.6871013 & 349.08687 \\
28 & 27.073574 & 12.926426 & 61.4714579 & 333.29957 \\
29 & 28.427253 & 11.572747 & 48.5450324 & 313.31562 \\
30 & 29.848616 & 10.151384 & 36.9722856 & 288.57597 \\
31 & 31.341047 & 8.658953 & 26.8209015 & 258.45777 \\
32 & 32.908099 & 7.091901 & 18.1619481 & 222.26760 \\
33 & 34.553504 & 5.446496 & 11.0700471 & 179.23383 \\
34 & 36.281179 & 3.718821 & 5.6235511 & 128.49829 \\
35 & 38.095238 & 1.904762 & 1.9047302 & 69.10701 \\
36 & 40.000000 & 0.000000 & -0.0000317 & 0.00000 \\
\end{longtable}

\hypertarget{b.}{%
\subsubsection{B.}\label{b.}}

Now consider extensions of the model in A. First, assume there is a
constant marginal cost of extraction c = 3.178025. The discount profits
in each year are now \(\pi_t=\rho^t (p_t-c)q_t\). Second, assume
marginal costs are zero again but the initial stock of reserves
increases to \(R_0\) = 948.2253 units. Third, assume marginal costs are
still zero and reserves go back to \(R_0\)=778.1259, but the demand
curve parameters are now \(a\)=40, \(b\)=0.75194. A lower value of \(b\)
means that the demand curve rotates outward around \(a\)=40.

List the values of \(t\), \(p_t\), \(q_t\), \(R_t\), and \(\pi_t\) for
the three cases.

{\textbf{Solution}}

\begin{Shaded}
\begin{Highlighting}[]
\NormalTok{exhaust\_b}\FloatTok{.1}\NormalTok{\_df }\OtherTok{\textless{}{-}} \FunctionTok{exhaust\_func}\NormalTok{(}\AttributeTok{a=}\DecValTok{40}\NormalTok{, }\AttributeTok{b=}\DecValTok{1}\NormalTok{, }\AttributeTok{R0=}\FloatTok{778.1259}\NormalTok{, }\AttributeTok{delta=}\FloatTok{0.05}\NormalTok{, }\AttributeTok{c=}\FloatTok{3.178025}\NormalTok{, }\AttributeTok{T=}\DecValTok{38}\NormalTok{)}

\FunctionTok{kable}\NormalTok{(exhaust\_b}\FloatTok{.1}\NormalTok{\_df)}
\end{Highlighting}
\end{Shaded}

\begin{longtable}[]{@{}rrrrr@{}}
\toprule\noalign{}
time & price & quantity & reserves & profits \\
\midrule\noalign{}
\endhead
\bottomrule\noalign{}
\endlastfoot
0 & 8.944544 & 31.055456 & 778.1259000 & 277.77689 \\
1 & 9.232870 & 30.767130 & 747.0704438 & 270.54182 \\
2 & 9.535612 & 30.464388 & 716.3033136 & 276.66341 \\
3 & 9.853491 & 30.146509 & 685.8389256 & 282.90320 \\
4 & 10.187265 & 29.812735 & 655.6924170 & 289.24783 \\
5 & 10.537727 & 29.462273 & 625.8796816 & 295.68132 \\
6 & 10.905712 & 29.094288 & 596.4174083 & 302.18469 \\
7 & 11.292096 & 28.707904 & 567.3231200 & 308.73563 \\
8 & 11.697800 & 28.302200 & 538.6152161 & 315.30807 \\
9 & 12.123788 & 27.876212 & 510.3130157 & 321.87171 \\
10 & 12.571077 & 27.428923 & 482.4368041 & 328.39152 \\
11 & 13.040729 & 26.959271 & 455.0078806 & 334.82719 \\
12 & 13.533864 & 26.466136 & 428.0486097 & 341.13247 \\
13 & 14.051656 & 25.948344 & 401.5824740 & 347.25448 \\
14 & 14.595338 & 25.404662 & 375.6341303 & 353.13298 \\
15 & 15.166203 & 24.833796 & 350.2294682 & 358.69944 \\
16 & 15.765612 & 24.234388 & 325.3956716 & 363.87615 \\
17 & 16.394992 & 23.605008 & 301.1612840 & 368.57516 \\
18 & 17.055840 & 22.944160 & 277.5562758 & 372.69707 \\
19 & 17.749731 & 22.250269 & 254.6121159 & 376.12980 \\
20 & 18.478316 & 21.521684 & 232.3618468 & 378.74712 \\
21 & 19.243331 & 20.756669 & 210.8401630 & 380.40710 \\
22 & 20.046596 & 19.953404 & 190.0834937 & 380.95031 \\
23 & 20.890025 & 19.109975 & 170.1300897 & 380.19796 \\
24 & 21.775624 & 18.224376 & 151.0201143 & 377.94967 \\
25 & 22.705504 & 17.294496 & 132.7957388 & 373.98119 \\
26 & 23.681878 & 16.318122 & 115.5012433 & 368.04169 \\
27 & 24.707071 & 15.292929 & 99.1831218 & 359.85093 \\
28 & 25.783524 & 14.216477 & 83.8901930 & 349.09605 \\
29 & 26.913798 & 13.086202 & 69.6737165 & 335.42799 \\
30 & 28.100587 & 11.899413 & 56.5875148 & 318.45761 \\
31 & 29.346715 & 10.653285 & 44.6881019 & 297.75135 \\
32 & 30.655150 & 9.344850 & 34.0348171 & 272.82646 \\
33 & 32.029006 & 7.970994 & 24.6899668 & 243.14573 \\
34 & 33.471555 & 6.528445 & 16.7189727 & 208.11163 \\
35 & 34.986232 & 5.013769 & 10.1905276 & 167.05987 \\
36 & 36.576642 & 3.423358 & 5.1767591 & 119.25233 \\
37 & 38.246573 & 1.753427 & 1.7534009 & 63.86913 \\
38 & 40.000000 & 0.000000 & -0.0000265 & 0.00000 \\
\end{longtable}

\begin{Shaded}
\begin{Highlighting}[]
\NormalTok{exhaust\_b}\FloatTok{.2}\NormalTok{\_df }\OtherTok{\textless{}{-}} \FunctionTok{exhaust\_func}\NormalTok{(}\AttributeTok{a=}\DecValTok{40}\NormalTok{, }\AttributeTok{b=}\DecValTok{1}\NormalTok{, }\AttributeTok{R0=}\FloatTok{948.2253}\NormalTok{, }\AttributeTok{delta=}\FloatTok{0.05}\NormalTok{, }\AttributeTok{c=}\DecValTok{0}\NormalTok{, }\AttributeTok{T=}\DecValTok{41}\NormalTok{)}

\FunctionTok{kable}\NormalTok{(exhaust\_b}\FloatTok{.2}\NormalTok{\_df)}
\end{Highlighting}
\end{Shaded}

\begin{longtable}[]{@{}rrrrr@{}}
\toprule\noalign{}
time & price & quantity & reserves & profits \\
\midrule\noalign{}
\endhead
\bottomrule\noalign{}
\endlastfoot
0 & 5.411264 & 34.588736 & 948.2253000 & 187.16878 \\
1 & 5.681827 & 34.318173 & 913.6365641 & 185.70470 \\
2 & 5.965919 & 34.034081 & 879.3183914 & 193.37577 \\
3 & 6.264215 & 33.735785 & 845.2843100 & 201.26495 \\
4 & 6.577425 & 33.422575 & 811.5485246 & 209.36618 \\
5 & 6.906297 & 33.093703 & 778.1259499 & 217.67136 \\
6 & 7.251611 & 32.748389 & 745.0322465 & 226.17008 \\
7 & 7.614192 & 32.385808 & 712.2838579 & 234.84930 \\
8 & 7.994902 & 32.005098 & 679.8980499 & 243.69296 \\
9 & 8.394647 & 31.605353 & 647.8929515 & 252.68169 \\
10 & 8.814379 & 31.185621 & 616.2875982 & 261.79227 \\
11 & 9.255098 & 30.744902 & 585.1019772 & 270.99722 \\
12 & 9.717853 & 30.282147 & 554.3570751 & 280.26424 \\
13 & 10.203746 & 29.796254 & 524.0749280 & 289.55562 \\
14 & 10.713933 & 29.286067 & 494.2786734 & 298.82758 \\
15 & 11.249629 & 28.750371 & 464.9926062 & 308.02954 \\
16 & 11.812111 & 28.187889 & 436.2422356 & 317.10331 \\
17 & 12.402716 & 27.597284 & 408.0543465 & 325.98217 \\
18 & 13.022852 & 26.977148 & 380.4570629 & 334.58991 \\
19 & 13.673995 & 26.326005 & 353.4799151 & 342.83967 \\
20 & 14.357695 & 25.642305 & 327.1539100 & 350.63275 \\
21 & 15.075579 & 24.924421 & 301.5116045 & 357.85722 \\
22 & 15.829358 & 24.170642 & 276.5871839 & 364.38643 \\
23 & 16.620826 & 23.379174 & 252.4165421 & 370.07732 \\
24 & 17.451867 & 22.548133 & 229.0373683 & 374.76859 \\
25 & 18.324461 & 21.675539 & 206.4892358 & 378.27864 \\
26 & 19.240684 & 20.759316 & 184.8136967 & 380.40328 \\
27 & 20.202718 & 19.797282 & 164.0543806 & 380.91324 \\
28 & 21.212854 & 18.787146 & 144.2570988 & 379.55141 \\
29 & 22.273497 & 17.726503 & 125.4699528 & 376.02973 \\
30 & 23.387172 & 16.612828 & 107.7434495 & 370.02578 \\
31 & 24.556530 & 15.443470 & 91.1306211 & 361.17908 \\
32 & 25.784357 & 14.215643 & 75.6871512 & 349.08687 \\
33 & 27.073574 & 12.926426 & 61.4715079 & 333.29957 \\
34 & 28.427253 & 11.572747 & 48.5450824 & 313.31562 \\
35 & 29.848616 & 10.151384 & 36.9723356 & 288.57597 \\
36 & 31.341047 & 8.658953 & 26.8209514 & 258.45777 \\
37 & 32.908099 & 7.091901 & 18.1619981 & 222.26760 \\
38 & 34.553504 & 5.446496 & 11.0700971 & 179.23383 \\
39 & 36.281179 & 3.718821 & 5.6236010 & 128.49829 \\
40 & 38.095238 & 1.904762 & 1.9047802 & 69.10701 \\
41 & 40.000000 & 0.000000 & 0.0000182 & 0.00000 \\
\end{longtable}

\begin{Shaded}
\begin{Highlighting}[]
\NormalTok{exhaust\_b}\FloatTok{.3}\NormalTok{\_df }\OtherTok{\textless{}{-}} \FunctionTok{exhaust\_func}\NormalTok{(}\AttributeTok{a=}\DecValTok{40}\NormalTok{, }\AttributeTok{b=}\FloatTok{0.75194}\NormalTok{, }\AttributeTok{R0=}\FloatTok{778.1259}\NormalTok{, }\AttributeTok{delta=}\FloatTok{0.05}\NormalTok{, }\AttributeTok{c=}\DecValTok{0}\NormalTok{, }\AttributeTok{T=}\DecValTok{30}\NormalTok{)}

\FunctionTok{kable}\NormalTok{(exhaust\_b}\FloatTok{.3}\NormalTok{\_df)}
\end{Highlighting}
\end{Shaded}

\begin{longtable}[]{@{}rrrrr@{}}
\toprule\noalign{}
time & price & quantity & reserves & profits \\
\midrule\noalign{}
\endhead
\bottomrule\noalign{}
\endlastfoot
0 & 9.255098 & 40.887441 & 778.1259000 & 378.41727 \\
1 & 9.717853 & 40.272026 & 737.2384594 & 372.72154 \\
2 & 10.203746 & 39.625841 & 696.9664335 & 385.07809 \\
3 & 10.713933 & 38.947346 & 657.3405930 & 397.40880 \\
4 & 11.249629 & 38.234927 & 618.3932472 & 409.64643 \\
5 & 11.812111 & 37.486886 & 580.1583207 & 421.71358 \\
6 & 12.402716 & 36.701444 & 542.6714346 & 433.52152 \\
7 & 13.022852 & 35.876729 & 505.9699909 & 444.96890 \\
8 & 13.673995 & 35.010779 & 470.0932617 & 455.94020 \\
9 & 14.357695 & 34.101531 & 435.0824827 & 466.30416 \\
10 & 15.075579 & 33.146821 & 400.9809514 & 475.91193 \\
11 & 15.829358 & 32.144376 & 367.8341303 & 484.59508 \\
12 & 16.620826 & 31.091808 & 335.6897548 & 492.16336 \\
13 & 17.451867 & 29.986611 & 304.5979472 & 498.40225 \\
14 & 18.324461 & 28.826155 & 274.6113359 & 503.07024 \\
15 & 19.240684 & 27.607676 & 245.7851807 & 505.89578 \\
16 & 20.202718 & 26.328273 & 218.1775045 & 506.57399 \\
17 & 21.212854 & 24.984900 & 191.8492311 & 504.76290 \\
18 & 22.273497 & 23.574359 & 166.8643307 & 500.07943 \\
19 & 23.387172 & 22.093290 & 143.2899721 & 492.09482 \\
20 & 24.556530 & 20.538168 & 121.1966821 & 480.32965 \\
21 & 25.784357 & 18.905289 & 100.6585144 & 464.24831 \\
22 & 27.073574 & 17.190767 & 81.7532249 & 443.25287 \\
23 & 28.427253 & 15.390519 & 64.5624577 & 416.67636 \\
24 & 29.848616 & 13.500258 & 49.1719388 & 383.77526 \\
25 & 31.341047 & 11.515484 & 35.6716806 & 343.72127 \\
26 & 32.908099 & 9.431472 & 24.1561962 & 295.59220 \\
27 & 34.553504 & 7.243259 & 14.7247242 & 238.36188 \\
28 & 36.281179 & 4.945635 & 7.4814654 & 170.88902 \\
29 & 38.095238 & 2.533130 & 2.5358303 & 91.90495 \\
30 & 40.000000 & 0.000000 & 0.0027001 & 0.00000 \\
\end{longtable}

\hypertarget{c.}{%
\subsubsection{C.}\label{c.}}

What is the final year \(T\) for the three cases? Explain why they are
different from part A.

{\textbf{The final year \(T\) for the tree cases is 38, 41, and 30
respectively. This is different from A because the parameters have been
altered. In the first scenario, the marginal cost is 3.178025 and in A
the marginal cost is 0, this resulted in T changing from 36 in A to 38.
When the marginal cost is no longer 0, the price increases, thus
decreasing demand slightly. Since demand decreases a little bit, it
takes the manager in this scenario a little longer to deplete their
resources, thus a slightly longer T value than in A.}}

{\textbf{In the second scenario, when R0 changes from 778.1259 to
948.2253, the price, quantity, and profits all decrease, but the
reserves increased. Since we start with higher initial reserves, it
takes longer to deplete the reserves, thus why T=41.}}

{ \textbf{Finally, in the last scenario, b decreases from 1 to 0.75194.
In this scenario, prices, quantities, and profits are the highest. This
decrease in elasticity (b) means that consumers are less responsive to
price changes. Since producers are profit maximizers, they will increase
prices since they know that consumers will continue to purchase it
despite the increased prices. Since producers can now sell the resources
for more money, they will produce more to maximize their profit. Thus,
why this scenario has the shortest T value. Producers quickly deplete
the resource in pursuit of maximum profits.}}

\hypertarget{d.}{%
\subsubsection{D.}\label{d.}}

Compare the path of prices in the three cases to that in part A. Explain
the differences.

{\textbf{In each scenario, the line travels till 40, which is a, our
choke point. However the price path changes depending on the parameters.
In Scenario A, the price path is not influenced by cost, and steadily
increases to the choke point.}}

{\textbf{When there is a cost of extraction, the price path is not as
steep as A because managers will want to extract at a rate to account
for their costs. The prices in this scenario are higher than in A in
order to cover the cost of extraction.}}

{\textbf{In the third scenario where this is no cost of extraction, but
a higher initial resource, the price curve is the lowest at T0. Since
there is such a large amount of initial reserves, the manager can price
it lower and have a larger quantity than in Scenario A.}}

{\textbf{Finally, in the scenario where b is lowered, the price path is
the steepest as a result of managers wanting to maximize their profit as
much as possible. As explained before, this is why it has the shortest T
value. Buyers are not sensitive to the increased price and continue to
buy, while producers continue to produce to meet demand, leading to a
steeper price path indicating early depletion. }}

\begin{Shaded}
\begin{Highlighting}[]
\NormalTok{price\_plot }\OtherTok{\textless{}{-}} \FunctionTok{ggplot}\NormalTok{() }\SpecialCharTok{+} 
  \FunctionTok{geom\_line}\NormalTok{(}\AttributeTok{data=}\NormalTok{exhaust\_a\_df, }\FunctionTok{aes}\NormalTok{(}\AttributeTok{x=}\NormalTok{time, }\AttributeTok{y=}\NormalTok{price, }\AttributeTok{color=}\StringTok{"Scenario A"}\NormalTok{), }\AttributeTok{show.legend =} \ConstantTok{TRUE}\NormalTok{) }\SpecialCharTok{+}
  \FunctionTok{geom\_line}\NormalTok{(}\AttributeTok{data=}\NormalTok{exhaust\_b}\FloatTok{.1}\NormalTok{\_df, }\FunctionTok{aes}\NormalTok{(}\AttributeTok{x=}\NormalTok{time, }\AttributeTok{y=}\NormalTok{price, }\AttributeTok{color=}\StringTok{"Scenario where c=3.178025"}\NormalTok{), }\AttributeTok{show.legend =} \ConstantTok{TRUE}\NormalTok{) }\SpecialCharTok{+}
  \FunctionTok{geom\_line}\NormalTok{(}\AttributeTok{data=}\NormalTok{exhaust\_b}\FloatTok{.2}\NormalTok{\_df, }\FunctionTok{aes}\NormalTok{(}\AttributeTok{x=}\NormalTok{time, }\AttributeTok{y=}\NormalTok{price, }\AttributeTok{color=}\StringTok{"Scenario where R0=948.2253"}\NormalTok{), }\AttributeTok{show.legend =} \ConstantTok{TRUE}\NormalTok{) }\SpecialCharTok{+}
  \FunctionTok{geom\_line}\NormalTok{(}\AttributeTok{data=}\NormalTok{exhaust\_b}\FloatTok{.3}\NormalTok{\_df, }\FunctionTok{aes}\NormalTok{(}\AttributeTok{x=}\NormalTok{time, }\AttributeTok{y=}\NormalTok{price, }\AttributeTok{color=}\StringTok{"Scenario where b=0.75194"}\NormalTok{), }\AttributeTok{show.legend =} \ConstantTok{TRUE}\NormalTok{)}\SpecialCharTok{+}
  \FunctionTok{labs}\NormalTok{(}\AttributeTok{reserves=}\StringTok{"Prices in different scenarios across T"}\NormalTok{, }\AttributeTok{x=}\StringTok{"Time"}\NormalTok{, }\AttributeTok{y=}\StringTok{"Price"}\NormalTok{) }\SpecialCharTok{+}
  \FunctionTok{scale\_color\_manual}\NormalTok{(}\AttributeTok{values=}\FunctionTok{c}\NormalTok{(}\StringTok{"Scenario A"}\OtherTok{=}\StringTok{"blue"}\NormalTok{, }\StringTok{"Scenario where c=3.178025"}\OtherTok{=}\StringTok{"red"}\NormalTok{, }\StringTok{"Scenario where R0=948.2253"}\OtherTok{=}\StringTok{"green"}\NormalTok{, }\StringTok{"Scenario where b=0.75194"}\OtherTok{=}\StringTok{"orange"}\NormalTok{)) }\SpecialCharTok{+}
  \FunctionTok{theme\_classic}\NormalTok{()}

\NormalTok{price\_plot}
\end{Highlighting}
\end{Shaded}

\includegraphics{ESM242_HW4_F23_files/figure-latex/unnamed-chunk-6-1.pdf}

\hypertarget{e.}{%
\subsubsection{E.}\label{e.}}

What happens to the present discounted value of profits (in other words,
the value of the reserves \(R_0\)) in the three cases compared to part
A? Explain the differences.

\textbf{In all scenarios, the present discounted value of profits
increase before declining as the reserves become depleted. All the
scenarios start with some amount of profit because there are reserves of
the exhaustible resource that can readily be extracted and sold. When
the present value of discounted profits decline, this is typically much
sharper than when they are increasing because the exhaustible resource
is growing smaller and smaller with no recharge. Managers are stuck in a
place of needing to make a profit, so they increase prices, and the
combined effect of less buyers as a result of the increase of price AND
declining reserves, their profits will decline much quicker than they
have been earning. In addition, since this is an exhaustible resource,
managers will continue to extract until there is none left, thus why
each curve ends at 0.}

{\textbf{In Scenario A, profits steadily increase before declining
around \(T_22\). The same happens for the scenario where the cost of
extraction is 3.178025. The difference between these two is that when
there is a cost of extraction, the present discounted value of profits
begin higher because of the increase in price to account for the cost of
extraction. In addition, this curve is more stretched out because T=38,
meaning that the present discounted value of profits is more spread out
across time. }}

{\textbf{In the scenario where R0 is higher, the profits start out lower
because of the lower price compared to scenario A, despite both not
having a cost of extraction. However, because of the larger initial
reserves, this scenario's present discounted value of profits increases
for a longer time than A simply because it has more initial reserves to
begin with. As stated previously, the larger amount of initial reserves
allows this scenario to have the longest time series before depletion.}}

{\textbf{Finally, the scenario where b is lowered has the most dramatic
increase and decline in present discounted value of profits. This is due
to a rapid depletion of resources in a scenario where demand is high and
inelastic.}}

\begin{Shaded}
\begin{Highlighting}[]
\NormalTok{profits\_plot }\OtherTok{\textless{}{-}} \FunctionTok{ggplot}\NormalTok{() }\SpecialCharTok{+} 
  \FunctionTok{geom\_line}\NormalTok{(}\AttributeTok{data=}\NormalTok{exhaust\_a\_df, }\FunctionTok{aes}\NormalTok{(}\AttributeTok{x=}\NormalTok{time, }\AttributeTok{y=}\NormalTok{profits, }\AttributeTok{color=}\StringTok{"Scenario A"}\NormalTok{), }\AttributeTok{show.legend =} \ConstantTok{TRUE}\NormalTok{) }\SpecialCharTok{+}
  \FunctionTok{geom\_line}\NormalTok{(}\AttributeTok{data=}\NormalTok{exhaust\_b}\FloatTok{.1}\NormalTok{\_df, }\FunctionTok{aes}\NormalTok{(}\AttributeTok{x=}\NormalTok{time, }\AttributeTok{y=}\NormalTok{profits, }\AttributeTok{color=}\StringTok{"Scenario where c=3.178025"}\NormalTok{), }\AttributeTok{show.legend =} \ConstantTok{TRUE}\NormalTok{) }\SpecialCharTok{+}
  \FunctionTok{geom\_line}\NormalTok{(}\AttributeTok{data=}\NormalTok{exhaust\_b}\FloatTok{.2}\NormalTok{\_df, }\FunctionTok{aes}\NormalTok{(}\AttributeTok{x=}\NormalTok{time, }\AttributeTok{y=}\NormalTok{profits, }\AttributeTok{color=}\StringTok{"Scenario where R0=948.2253"}\NormalTok{), }\AttributeTok{show.legend =} \ConstantTok{TRUE}\NormalTok{) }\SpecialCharTok{+}
  \FunctionTok{geom\_line}\NormalTok{(}\AttributeTok{data=}\NormalTok{exhaust\_b}\FloatTok{.3}\NormalTok{\_df, }\FunctionTok{aes}\NormalTok{(}\AttributeTok{x=}\NormalTok{time, }\AttributeTok{y=}\NormalTok{profits, }\AttributeTok{color=}\StringTok{"Scenario where b=0.75194"}\NormalTok{), }\AttributeTok{show.legend =} \ConstantTok{TRUE}\NormalTok{)}\SpecialCharTok{+}
  \FunctionTok{labs}\NormalTok{(}\AttributeTok{title=}\StringTok{"Present discounted profits in different scenarios across T"}\NormalTok{, }\AttributeTok{x=}\StringTok{"Time"}\NormalTok{, }\AttributeTok{y=}\StringTok{"Profits"}\NormalTok{) }\SpecialCharTok{+}
  \FunctionTok{scale\_color\_manual}\NormalTok{(}\AttributeTok{values=}\FunctionTok{c}\NormalTok{(}\StringTok{"Scenario A"}\OtherTok{=}\StringTok{"blue"}\NormalTok{, }\StringTok{"Scenario where c=3.178025"}\OtherTok{=}\StringTok{"red"}\NormalTok{, }\StringTok{"Scenario where R0=948.2253"}\OtherTok{=}\StringTok{"green"}\NormalTok{, }\StringTok{"Scenario where b=0.75194"}\OtherTok{=}\StringTok{"orange"}\NormalTok{)) }\SpecialCharTok{+}
  \FunctionTok{theme\_classic}\NormalTok{()}

\NormalTok{profits\_plot}
\end{Highlighting}
\end{Shaded}

\includegraphics{ESM242_HW4_F23_files/figure-latex/unnamed-chunk-7-1.pdf}

\end{document}
